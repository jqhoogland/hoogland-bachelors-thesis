In the age of \textit{big data}, machine learning (ML), a subset of
artificial intelligence (AI), has become more than \textit{just}
another set of data analysis tools~\cite{deep-learning}. For one, ML's
connections with theoretical physics are multivarious and deeply
conceptual; the very success of ML may, in part, result from physical
principles including symmetry, locality, and
hierarchy~\cite{lin}. Furthermore, ML and theoretical physics share a
powerful conceptual framework in information
theory~\cite{tishby}. Beyond data analysis, the intersection of ML and
physics contains a unique set of ideas that researchers in both fields
can leverage to solve tough problems.

In particular, recent work has drawn attention to the similarities
between ML and a class of techniques from statistical physics known as
the renormalization group (RG)~\cite{mehta,kjr,iso,lin}. Developed in
the last century, RG has been crucial in making sense of critical
behavior, those phenomena characterizing phase transitions. In 2014,
Mehta and Schwab published a seminal paper describing an
\textit{exact} equivalence between a technique from RG and a type of
neural network (NN) from ML~\cite{mehta}. This, however, met
criticism, and it works only under a narrow set of circumstances. The
similarities between ML and RG, then, are still largely qualitative,
and this remains an active area of research. Not to mention, the
research landscape maintains lingering misconceptions about the
details of the intersection~\cite{iso, lin, mehta-reply, hoeve}. This
warrants further investigation, and in order to facilitate and
encourage such research, our first contribution is to provide a
clarifying overview of the competing views. We resolve a number of
inaccuracies.

In 2018, Koch-Janusz and Ringel derived an algorithm which uses neural
networks to learn RG transformations on lattice systems: the
\textit{real-space mutual information} (RSMI)
algorithm~\cite{kjr}. Notably, this method is \textit{unsupervised}
which is particularly relevant for research into poorly understood
physical systems; information-theoretic approaches, like this
algorithm, may guide researchers towards the locations of critical
points and even calculations of critical exponents. Furthermore,
Koch-Janusz and Ringel's derivation is \textit{optimal} in a rigorous
sense~\cite{kjr,lenggenhager}. This is
exciting because many well-established practices in RG lack precise
justification. More exact formulations, like these, may inspire more
effective implementations, not to mention a better understanding of
why these ML and RG techniques work.

In our investigation, we will develop a set of tools for tackling
critical phenomena. First, we consider some of the standard techniques
of statistical physics, building towards an ML-derived
implementation of RG.\@ We, then, introduce elements
of ML, emphasizing their utility in a variety of statistical physics
contexts. We anchor this investigation around the Ising
model, one of the most important models in statistical physics. To
compare these various techniques, we evaluate their ability in
predicting the Ising model's correlation length critical exponent,
$\nu$.

To accomplish this, we have built and shared an open-source
implementation of the RSMI algorithm~\cite{rgpy} through
\textit{rgpy}. Hereby, we provide a calculation
for $\nu$.Then, we describe a generalization of
this algorithm to \textit{arbitrary} lattice systems. This gives rise
to a family of RSMI-inspired approaches.

It is our aim to enable and inspire researchers to build further on
our results. We accomplish this by reviewing the current state of
research, sharing an implementation of the RSMI algorithm, and
describing avenues of future research. Although we focus on the
perspective of statistical physicists, this capstone is accessible for
both ML and physics researchers, even at an undergraduate level.

In chapter 2, we begin by introducing techniques native to
statistical physics. We describe mean-field theory, and its failures
bring us to the renormalization group. In chapter 3, we discuss
two examples of ML in physical investigations.  First, we use neural
networks to classify Ising model phases. Then, we use the same neural
networks to generate new samples of Ising models. These examples serve
to introduce the basics of ML, assuming no prior knowledge (except
mathematical maturity), and the same is true for the portion on
statistical physics. In chapter 4, we explore the
similarities between ML and RG, and by being explicit in our
formulation of ``relevant'' information, we manage to avoid some of
the mistakes of earlier comparisons. In chapter 5, we explain
and justify the RSMI algorithm, following the formulation of
Koch-Janusz and Ringel. In chapter 6, we provide our own
results: a recalculation of $\nu$ and a generalization of this
technique, paving the way for a new class of RG techniques. In
chapter 7, we close with a discussion, reflecting on our
comparisons of ML and statistical physics and emphasizing the
wide-ranging impacts of these ideas.
